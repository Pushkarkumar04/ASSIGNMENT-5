\documentclass{article}
\usepackage[utf8]{inputenc}
\usepackage{graphicx}

\title{ASSIGNMENT 5\\ON\\Emerging Technologies In Healthcare​​}
\author{BY\\PUSHKAR KUMAR\\ROLL NO.: 21111040\\FIRST SEMESTER\\BRANCH:BIOMEDICAL ENGINEERING\\SECTION:A\\NATIONAL INSTITUTE OF TECHNOLOGY, RAIPUR\\ASSIGNMENT SUBMITTED TO\\
DEPARTMENT OF BIOMEDICAL ENGINEERING}
\date{}

\begin{document}

\maketitle
\begin{figure}[h]
    \centering
    \includegraphics[height=9cm,width=9cm]{download.jpg}
\end{figure}


\section{Emerging Technologies in Healthcare} 
In today’s connected world, technology is becoming increasingly relevant in every business industry, as well as our personal lives. Of all the industries that technology plays a key role in, healthcare technology is certainly one of the most critical. Care providers and organizations are relying evermore on emerging technology in healthcare for improving and saving countless lives worldwide.There’s never been a more exciting time to be in the digital healthcare space than right now.Health technologies encompass all the devices, medicines, vaccines, procedures and systems designed to streamline healthcare operations, lower costs and enhance the quality of care. In 2020 and 2021, the Covid-19 pandemic forced healthcare into the future, and, as a result, several promising medical technologies were tested on a massive scale. In 2022, the question is how those technologies can be used together in a post-pandemic world.
\subsection{ REMOTE PATIENT MONITORING}
Remote patient monitoring, also referred to as remote physiologic monitoring, is the use of digital technologies to monitor and capture medical and other health data from patients and electronically transmit this information to healthcare providers for assessment and, when necessary, recommendations and instructions.When the pandemic hit, the value of providing remote patient monitoring services to patients who were expected to reduce travel and direct contact with others became even more apparent.Thanks to remote patient monitoring (RPM), physicians can now know what is going on with a patient without physically being close. There are several benefits to RPM including better patient outcomes, faster response time, and significant cost reductions over time. In fact, RPM goes hand in hand with telemedicine in reducing the need for patient travel and mitigating everyone’s exposure.Thanks to legislative changes to Medicare for the Covid-19 pandemic, various forms of RPM were approved for reimbursement, effectively increasing the popularity of this new technology.
\subsection{ARTIFICIAL INTELLIGENCE}
Artificial intelligence (AI) is the ability of a computer or a robot controlled by a computer to do tasks that are usually done by humans because they require human intelligence and discernment.Artificial intelligence (AI) takes on many different forms in healthcare. The primary trend for AI in healthcare 2022 will be in utilizing machine learning to evaluate large amounts of patient data and other information. By creating tailored algorithms, programmers can mimic human thought and write programs that can seemingly think, learn, make decisions, and take action.No, this does not mean that medical care will suddenly be delivered by sentient robots. However, it does mean that, given a patient’s particular medical records, history, and current symptoms, physicians may be given suggested diagnoses, medications, and treatments plans. The physicians will always have the final say, but the information will be at their disposal.
\subsection{DIGITAL THERAPEUTICS}
Digital therapeutics, a subset of digital health, are evidence-based therapeutic interventions driven by high quality software programs to prevent, manage, or treat a medical disorder or disease.Patients that have chronic illnesses often require ongoing care from their physicians. This care can include patient education, symptom monitoring, medication adjustment, and behavioral changes. Not only is this care costly, but it is also very time-consuming for both medical staff and patients. Now, there are new digital therapeutics that can fill this role.Digital therapeutics are prescribed by a doctor to a patient for their particular medical condition. These sophisticated software programs can be accessed as apps on a patient’s smartphone or through a personal computer. They go through the same rigorous testing as all medications, including randomized clinical trials.As patients use the applications, information about their wellbeing is reported back to their physician. This allows doctors to be able to monitor patients without having to see them regularly, as well as spot problems much earlier than when a patient needs to wait for an appointment.
\subsection{TECHNOLOGY IN MENTAL HEALTH}
There are several new technologies that have emerged over the past year that can help address a patient’s ongoing mental health needs. While most assessments and initial treatments may still need to be completed by a clinician, there are now additional tools patients can use to improve their mental health between appointments.
Mental health technologies used by professionals as an adjunct to mainstream clinical practices include email, SMS, virtual reality, computer programs, blogs, social networks, the telephone, video conferencing, computer games, instant messaging and podcasts.
\subsection{3D BIOPRINTING}
The invention of 3D printing is another new technology in the healthcare industry that is proving to be transformative. This new field of 3D Bioprinting enables physicians to print artificial limbs, organs, joint replacement parts, and bio tissues. In addition, in the field of pharmacology, there are ongoing experiments for printing pills and other medications. Lastly, 3D printers can also help create medical devices and surgical tools.
\subsection{ WEARABLES WITH A PURPOSE}
Fitness trackers have been on the rise for years: FitBit shipped 9.9 million of its wearable devices in 2019. But the next trend in wearables for medical technology is more specific. For diabetes patients, wearable continuous glucose monitors (CGMs) are set to become the new normal.Wearable CGMs remove the need for intermittent glucose testing and instead keep track of one’s blood sugar levels in real time. This allows users to see the immediate impacts of food and exercise, and shape their lifestyles accordingly. 




\end{document}